% Options for packages loaded elsewhere
\PassOptionsToPackage{unicode}{hyperref}
\PassOptionsToPackage{hyphens}{url}
\PassOptionsToPackage{dvipsnames,svgnames,x11names}{xcolor}
%
\documentclass[
  11pt,
  a4paper,
]{article}

\usepackage{amsmath,amssymb}
\usepackage{setspace}
\usepackage{iftex}
\ifPDFTeX
  \usepackage[T1]{fontenc}
  \usepackage[utf8]{inputenc}
  \usepackage{textcomp} % provide euro and other symbols
\else % if luatex or xetex
  \usepackage{unicode-math}
  \defaultfontfeatures{Scale=MatchLowercase}
  \defaultfontfeatures[\rmfamily]{Ligatures=TeX,Scale=1}
\fi
\usepackage{lmodern}
\ifPDFTeX\else  
    % xetex/luatex font selection
\fi
% Use upquote if available, for straight quotes in verbatim environments
\IfFileExists{upquote.sty}{\usepackage{upquote}}{}
\IfFileExists{microtype.sty}{% use microtype if available
  \usepackage[]{microtype}
  \UseMicrotypeSet[protrusion]{basicmath} % disable protrusion for tt fonts
}{}
\makeatletter
\@ifundefined{KOMAClassName}{% if non-KOMA class
  \IfFileExists{parskip.sty}{%
    \usepackage{parskip}
  }{% else
    \setlength{\parindent}{0pt}
    \setlength{\parskip}{6pt plus 2pt minus 1pt}}
}{% if KOMA class
  \KOMAoptions{parskip=half}}
\makeatother
\usepackage{xcolor}
\usepackage[top=2.5cm,bottom=2.5cm,left=2.5cm,right=2.5cm]{geometry}
\setlength{\emergencystretch}{3em} % prevent overfull lines
\setcounter{secnumdepth}{2}


\providecommand{\tightlist}{%
  \setlength{\itemsep}{0pt}\setlength{\parskip}{0pt}}\usepackage{longtable,booktabs,array}
\usepackage{calc} % for calculating minipage widths
% Correct order of tables after \paragraph or \subparagraph
\usepackage{etoolbox}
\makeatletter
\patchcmd\longtable{\par}{\if@noskipsec\mbox{}\fi\par}{}{}
\makeatother
% Allow footnotes in longtable head/foot
\IfFileExists{footnotehyper.sty}{\usepackage{footnotehyper}}{\usepackage{footnote}}
\makesavenoteenv{longtable}
\usepackage{graphicx}
\makeatletter
\def\maxwidth{\ifdim\Gin@nat@width>\linewidth\linewidth\else\Gin@nat@width\fi}
\def\maxheight{\ifdim\Gin@nat@height>\textheight\textheight\else\Gin@nat@height\fi}
\makeatother
% Scale images if necessary, so that they will not overflow the page
% margins by default, and it is still possible to overwrite the defaults
% using explicit options in \includegraphics[width, height, ...]{}
\setkeys{Gin}{width=\maxwidth,height=\maxheight,keepaspectratio}
% Set default figure placement to htbp
\makeatletter
\def\fps@figure{htbp}
\makeatother

\usepackage[section]{placeins}
\usepackage{orcidlink}
\definecolor{mypink}{RGB}{219, 48, 122}
\makeatletter
\@ifpackageloaded{caption}{}{\usepackage{caption}}
\AtBeginDocument{%
\ifdefined\contentsname
  \renewcommand*\contentsname{Table of contents}
\else
  \newcommand\contentsname{Table of contents}
\fi
\ifdefined\listfigurename
  \renewcommand*\listfigurename{List of Figures}
\else
  \newcommand\listfigurename{List of Figures}
\fi
\ifdefined\listtablename
  \renewcommand*\listtablename{List of Tables}
\else
  \newcommand\listtablename{List of Tables}
\fi
\ifdefined\figurename
  \renewcommand*\figurename{Figure}
\else
  \newcommand\figurename{Figure}
\fi
\ifdefined\tablename
  \renewcommand*\tablename{Table}
\else
  \newcommand\tablename{Table}
\fi
}
\@ifpackageloaded{float}{}{\usepackage{float}}
\floatstyle{ruled}
\@ifundefined{c@chapter}{\newfloat{codelisting}{h}{lop}}{\newfloat{codelisting}{h}{lop}[chapter]}
\floatname{codelisting}{Listing}
\newcommand*\listoflistings{\listof{codelisting}{List of Listings}}
\makeatother
\makeatletter
\makeatother
\makeatletter
\@ifpackageloaded{caption}{}{\usepackage{caption}}
\@ifpackageloaded{subcaption}{}{\usepackage{subcaption}}
\makeatother
\ifLuaTeX
  \usepackage{selnolig}  % disable illegal ligatures
\fi
\usepackage[style=authoryear-comp,]{biblatex}
\addbibresource{references.bib}
\usepackage{bookmark}

\IfFileExists{xurl.sty}{\usepackage{xurl}}{} % add URL line breaks if available
\urlstyle{same} % disable monospaced font for URLs
\hypersetup{
  pdftitle={Forecasting interrupted time series},
  pdfauthor={Rob J Hyndman; Bahman Rostami-Tabar},
  pdfkeywords={Forecasting, interrupted time series, disruptive events,
COVID-19},
  colorlinks=true,
  linkcolor={blue},
  filecolor={Maroon},
  citecolor={Blue},
  urlcolor={Blue},
  pdfcreator={LaTeX via pandoc}}

%% CAPTIONS
\usepackage{caption}
\DeclareCaptionStyle{italic}[justification=centering]
 {labelfont={bf},textfont={it},labelsep=colon}
\captionsetup[figure]{style=italic,format=hang,singlelinecheck=true}
\captionsetup[table]{style=italic,format=hang,singlelinecheck=true}

%% FONT
\usepackage{bera}
\usepackage[charter]{mathdesign}
\usepackage[scale=0.9]{sourcecodepro}
\usepackage[lf,t]{FiraSans}

%% HEADERS AND FOOTERS
\usepackage{fancyhdr}
\pagestyle{fancy}
\rfoot{\Large\sffamily\raisebox{-0.1cm}{\textbf{\thepage}}}
\makeatletter
\lhead{\textsf{\expandafter{\@title}}}
\makeatother
\rhead{}
\cfoot{}
\setlength{\headheight}{15pt}
\renewcommand{\headrulewidth}{0.4pt}
\renewcommand{\footrulewidth}{0.4pt}
\fancypagestyle{plain}{%
\fancyhf{} % clear all header and footer fields
\fancyfoot[C]{\sffamily\thepage} % except the center
\renewcommand{\headrulewidth}{0pt}
\renewcommand{\footrulewidth}{0pt}}

%% MATHS
\usepackage{bm,amsmath}
\allowdisplaybreaks

%% GRAPHICS
\makeatletter
\def\fps@figure{htbp}
\makeatother
\setcounter{topnumber}{2}
\setcounter{bottomnumber}{2}
\setcounter{totalnumber}{4}
\renewcommand{\topfraction}{0.85}
\renewcommand{\bottomfraction}{0.85}
\renewcommand{\textfraction}{0.15}
\renewcommand{\floatpagefraction}{0.8}

%% SECTION TITLES
\usepackage[compact,sf,bf]{titlesec}
\titleformat{\section}[block]
  {\fontsize{15}{17}\bfseries\sffamily}
  {\thesection}
  {0.4em}{}
\titleformat{\subsection}[block]
  {\fontsize{12}{14}\bfseries\sffamily}
  {\thesubsection}
  {0.4em}{}
\titlespacing{\section}{0pt}{*5}{*1}
\titlespacing{\subsection}{0pt}{*2}{*0.2}

%% BIBLIOGRAPHY.

\makeatletter
\@ifpackageloaded{biblatex}{
\ExecuteBibliographyOptions{bibencoding=utf8,minnames=1,maxnames=3, maxbibnames=99,dashed=false,terseinits=true,giveninits=true,uniquename=false,uniquelist=false,doi=false, isbn=false,url=true,sortcites=false}
\DeclareFieldFormat{url}{\texttt{\url{#1}}}
\DeclareFieldFormat[article]{pages}{#1}
\DeclareFieldFormat[inproceedings]{pages}{\lowercase{pp.}#1}
\DeclareFieldFormat[incollection]{pages}{\lowercase{pp.}#1}
\DeclareFieldFormat[article]{volume}{\mkbibbold{#1}}
\DeclareFieldFormat[article]{number}{\mkbibparens{#1}}
\DeclareFieldFormat[article]{title}{\MakeCapital{#1}}
\DeclareFieldFormat[article]{url}{}
\DeclareFieldFormat[inproceedings]{title}{#1}
\DeclareFieldFormat{shorthandwidth}{#1}
\usepackage{xpatch}
\xpatchbibmacro{volume+number+eid}{\setunit*{\adddot}}{}{}{}
% Remove In: for an article.
\renewbibmacro{in:}{%
  \ifentrytype{article}{}{%
  \printtext{\bibstring{in}\intitlepunct}}}
\AtEveryBibitem{\clearfield{month}}
\AtEveryCitekey{\clearfield{month}}
\DeclareDelimFormat[cbx@textcite]{nameyeardelim}{\addspace}
\renewcommand*{\finalnamedelim}{\addspace\&\space}
}{}
\makeatother% Placement of logos

\RequirePackage[absolute,overlay]{textpos}
\setlength{\TPHorizModule}{1cm}
\setlength{\TPVertModule}{1cm}
\def\placefig#1#2#3#4{\begin{textblock}{.1}(#1,#2)\rlap{\includegraphics[#3]{#4}}\end{textblock}}

% Title and date

\title{Forecasting interrupted time series}
\date{19 February 2024}

\def\Date{\number\day}
\def\Month{\ifcase\month\or
 January\or February\or March\or April\or May\or June\or
 July\or August\or September\or October\or November\or December\fi}
\def\Year{\number\year}

% Working paper number and JEL codes

\makeatletter
\def\wp#1{\gdef\@wp{#1}}\def\@wp{??/??}
\def\jel#1{\gdef\@jel{#1}}\def\@jel{??}
\def\showjel{{\large\textsf{\textbf{JEL classification:}}~\@jel}}
\def\nojel{\def\showjel{}}
\makeatother

\wp{no/yr}
\jel{C10,C14,C22}

% Title page

\makeatletter
\def\cover{{\sffamily\setcounter{page}{0}
        \thispagestyle{empty}
        \placefig{2}{1.5}{width=5cm}{monash2}
        \placefig{16.9}{1.5}{width=2.1cm}{MBSportrait}
        \begin{textblock}{4}(16.9,4)ISSN 1440-771X\end{textblock}
        \begin{textblock}{7}(12.7,27.9)\hfill
        \includegraphics[height=0.7cm]{AACSB}~~~
        \includegraphics[height=0.7cm]{EQUIS}~~~
        \includegraphics[height=0.7cm]{AMBA}
        \end{textblock}
        \vspace*{2.5cm}
        \begin{center}\Large
        Department of Econometrics and Business Statistics\\[.5cm]
        \footnotesize http://monash.edu/business/ebs/research/publications
        \end{center}\vspace{2cm}
        \begin{center}
        \fbox{\parbox{14cm}{\begin{onehalfspace}\centering\Huge\vspace*{0.3cm}
                \textsf{\textbf{\expandafter{\@title}}}\vspace{1cm}\par
                \LARGE
                \expandafter{\@author}
                \end{onehalfspace}
        }}
        \end{center}
        \vfill
                \begin{center}\Large
                \Month~\Year\\[1cm]
                Working Paper \@wp
        \end{center}\vspace*{2cm}}}
        \def\addresses#1{\gdef\@addresses{#1}}\def\@addresses{??}
        \def\pageone{{\sffamily\setstretch{1}%
        \thispagestyle{empty}%
        \vbox to \textheight{%
        \raggedright\baselineskip=1.2cm
     {\fontsize{24.88}{30}\sffamily\textbf{\expandafter{\@title}}}
        \vspace{2cm}\par
        \hspace{1cm}\parbox{14cm}{\sffamily\large\@addresses}\vspace{1cm}\vfill
        \hspace{1cm}{\large\Date~\Month~\Year}\\[1cm]
        \hspace{1cm}\showjel\vss}}}
\def\blindtitle{{\sffamily
     \thispagestyle{plain}\raggedright\baselineskip=1.2cm
     {\fontsize{24.88}{30}\sffamily\textbf{\expandafter{\@title}}}\vspace{1cm}\par
        }}
\def\titlepage{{\cover\newpage\pageone\newpage\blindtitle}}

\def\blind{\def\titlepage{{\blindtitle}}\let\maketitle\blindtitle}
\def\titlepageonly{\def\titlepage{{\pageone\end{document}}}}
\def\nocover{\def\titlepage{{\pageone\newpage\blindtitle}}\let\maketitle\titlepage}
\let\maketitle\titlepage
\makeatother

% Authors

\nocover
  \author{Rob J Hyndman, Bahman Rostami-Tabar}
  \addresses{%
    %
      \textbf{Rob J Hyndman}\\%
      %
        Department of Econometrics \& Business Statistics\\%
        %
        %
        %
      %
        %
        Monash University\\%
        %
        %
      %
        %
        %
        Clayton VIC \\%
        %
      %
        %
        %
        %
        Australia\\%
      %
        %
        %
        %
        %
      %
      {Email: Rob.Hyndman@monash.edu}\\%
      \textit{Corresponding author}\\[0.5cm]%
   %
      \textbf{Bahman Rostami-Tabar}\\%
      %
        %
        Cardiff Business School\\%
        %
        %
      %
        %
        %
        %
        %
      %
        %
        %
        %
        United Kingdom\\%
      %
      {Email: rostami-tabarb@cardiff.ac.uk}\\%
      \\[0.5cm]%
   %
   }%
   \lfoot{\sf Hyndman, Rostami-Tabar: 19 February 2024}

% Keywords

\newenvironment{keywords}{\par\vspace{0.5cm}\noindent{\sffamily\textbf{Keywords:}}}{\vspace{0.25cm}\par\hrule\vspace{0.5cm}\par}

% Abstract
\renewenvironment{abstract}{\begin{minipage}{\textwidth}\parskip=1.4ex\noindent
\hrule\vspace{0.1cm}\par{\sffamily\textbf{\abstractname}}\newline\setstretch{1.5}}
  {\end{minipage}}
\begin{document}
\maketitle

\begin{abstract}
Forecasting interrupted time series data is a major challenge for
forecasting teams, especially in light of events such as the COVID-19
pandemic. This paper investigates several strategies for dealing with
interruptions in time series forecasting, including highly adaptable
models, intervention models, marking interrupted periods as missing,
forecasting what may have been, downweighting the interruption period,
and ensemble models. Each approach offers specific advantages and
disadvantages, such as adaptability, memory retention, data integrity,
flexibility, and accuracy. We evaluate the effectiveness of these
strategies using two actual datasets that were interrupted by COVID-19,
and we provide recommendations for how to handle these interruptions.
This work contributes to the literature on time series forecasting,
offering insights for academics and practitioners dealing with
interrupted data in numerous domains.
\end{abstract}

\begin{keywords}
  Forecasting, interrupted time series, disruptive events, COVID-19; 
  Forecasting, interrupted time series, disruptive events, COVID-19
\end{keywords}

\setstretch{1.5}
\section{Introduction}\label{introduction}

Time series forecasting models use historical data to estimate future
values \autocite{fildes2008forecasting} based on consistent patterns
observed in the past. However, time series are sometimes disrupted by
unusual events that jeopardise the regularity of the time series
pattern. This can be defined as an \emph{interrupted time series}. Time
series data may be disrupted by a variety of reasons, ranging from
temporary policy changes to natural disasters. Outbreaks, epidemics, and
pandemics, for example, may have a considerable impact on population
behavior, causing discrepancies and interruptions of data in a variety
of industries such as healthcare, pharmaceuticals, transportation,
retail, tourism, and traffic management. Similarly, supply chain
disruptions induced by factors such as strikes or shortages cause
irregularities in sales, inventory levels, production, and delivery,
often resulting in data gaps. Transportation-related data may be
interrupted due to equipment failures, such as malfunctions or
breakdowns in vehicles, aircraft, or rail systems, resulting in missing
or inaccurate data points. Furthermore, cybersecurity breaches data
quality by possibly skewing readings in datasets relating to financial
transactions, network traffic, and user activity, complicating time
series analysis.

The disruptions may result in relatively simple changes in the series;
for example a level shift at the start of the disruption, and another at
the end of the disruption. Or they may be more complex, with changes to
the seasonal patterns, and changes to the level, which evolve over time.

The presence of such disrupted events poses a significant challenge to
time series forecasting approaches, restricting their ability to
reliably capture systematic information and forecast future values. The
implications of these temporal interruptions are far-reaching: time
series forecasting techniques, which are a key component of the
forecaster's toolbox, may no longer be viable options since time series
models usually assume that the data will evolve in the future in a
similar way to how it has evolved in the past. But a big event (such as
COVID-19) can result in a future that is different from the past, at
least in the short term. Therefore, practitioners face a critical
question: how to effectively forecast time series data that are impacted
by such disruptive events? Many forecasters have faced this issue
recently with the COVID-19 pandemic, where historical patterns have been
severely disrupted due to policy interventions, lockdowns, and other
restrictions. This study is motivated by numerous conversations with
academics and practitioners, which demonstrate widespread concern about
dealing with data affected by events like the COVID-19 pandemic. In this
paper, we consider a range of models that can be used to forecast time
series influenced by disrupted events, and compare their performance on
some real data sets.

It is important to note that in this study, we consider the problem of
forecasting interrupted time series both during and after the event.
This is a separate problem from change point detection
\autocite{truong2020selective} and anomaly detection
\autocite{talagala2020anomaly}. In the situations we investigate, we
know that a change has occurred which has caused some unusual
observations, and we want to forecast what will happen next. Change
point detection refers to determining \emph{when} a change occurred
\autocite{blazquez2021review}, and anomaly detection aims to identify
unusual observations with the intention of either minimising their
value, or paying particular attention to them, and doing a root-cause
investigation. Our study focus also differs from \emph{intervention
modeling}, sometimes referred to as \emph{interrupted time series
analysis and modelling} \autocite{mcdowall2019interrupted}, which is
used when we have data about an outcome over time and the aim is to
assess the impact of an intervention, policy, or program implemented at
a specific time point. We are not trying to measure the impact of the
interruption; rather we are trying to produce sensible forecasts during
and after the interruption.

To our knowledge, this is the first study to describe and compare
general strategies for forecasting interrupted time series, such as
COVID-19 affected data. The study proposes and evaluates multiple
forecasting strategies designed to handle interruptions in time series
forecasting caused by disruptive events. These approaches include a
range of models capable of capturing different types of changes, from
simple level shifts to more complex alterations in seasonal patterns and
trends. By comparing the performance of these models on two real data
sets affected by the COVID-19 pandemic, the paper offers valuable
insights into the effectiveness and practical implications of each
approach. By discussing the advantages and disadvantages of each
strategy, and providing recommendations based on the findings, the paper
equips practitioners with actionable insights for addressing the
challenges associated with forecasting during and after disruptive
events. Further, by adhering to reproducibility principles and providing
both data and R code for the forecasting models, the research
contributes to transparency and promotes the generalizability of the
suggested strategies to a variety of domains. This openness allows for
the replication of results, ensuring the reliability of findings, and
enhancing the accessibility and utility of study outputs for the larger
research community.

We introduce several strategies to handling interruptions when
forecasting in Section~\ref{sec-methods}. We then apply these approaches
to two real data sets in \textbf{?@sec-sec-examples}. Finally, in
\textbf{?@sec-conclusions}, we discuss some of the advantages and
disadvantages of each approach, and provide some recommendations for
practitioners.

\section{Related works}\label{related-works}

The presence of unusual observations in time series analysis is a
significant concern that has received attention in numerous studies. In
this section, we provide a brief overview of some of these relevant
publications.

One modelling approach that has gained popularity is the intervention
model, also known as interrupted time series (ITS) modelling, introduced
by \textcite{box1975intervention}. This approach focuses on
understanding how and whether outcomes change following the
implementation of an intervention, policy, or programme at a specific
time point. ITS models provide a robust tool for analysing policy and
programme evaluation across diverse domains
\autocite{bernal2017interrupted,mcdowall2019interrupted}. These models
can forecast future values by developing models that incorporate a range
of components, such as pre-intervention levels, trends, seasonality,
covariates related to the intervention time, and other external factors
that might affect the outcome variable. A related approach is the
multiple regimes model, which is particularly useful in forecasting due
to its ability to capture structural changes and transitions observed in
interrupted time series data \autocite{Koopg2007}.

There are also statistical models that are used to detect changes in the
time series data and that can model unusual observation and have proven
effective in similar situations. A range of methods such as simple
location shift and scale shift models \autocite{aue2013structural},
Bayesian \autocite{barry1993bayesian}, and supervised and supervised
machine learning \autocite{aminikhanghahi2017survey}. Facebook's Prophet
\autocite{taylor2018forecasting} is an example of a time series
forecasting method that can incorporate changepoint detection
(automatically or manually), and adjust model parameters to capture the
newest trends in a time series. Anomaly detection is another related
area in time series analysis. In the context of anomaly detection, time
series anomalies are defined as points or subsequences with unexpected
or abnormal values, and the detection goal is to identify such anomalies
\autocite{blazquez2021review}.

In the context of forecasting disruptions such as the COVID-19 pandemic,
Bayesian approaches have been proposed for detecting changing points in
outbreaks and forecasting case numbers under counterfactual scenarios
\autocite{dehning2020inferring}. Counterfactual analysis examines what
would have happened to the forecast variable of interest if the
disruptive event had not occurred.
\textcite{athanasopoulos2023probabilistic} proposed a variant of this
approach for forecasting tourism recovery from COVID-19, combining
forecast reconciliation and forecast combinations applied to historical
data, to generate COVID-free counterfactual forecasts of what might have
been if the pandemic never occurred. Then scenario-based judgemental
probabilistic forecasts were compared with the counterfactual forecasts,
to better understand the future recovery of the tourism sector from the
pandemic.

Inventory stockouts are a common cause of data interruption in the
supply chain. To deal with these interruptions, a popular technique is
to provide an estimate for the observation corresponding to the
stock-out period as if there were no stock-outs, and then generate
projections appropriately. \textcite{Bell2000} presented an adjustment
strategy for stock-out periods that involves smoothing demand volatility
and correcting for stockouts using predicted variance conditioned on the
observed stockout. \textcite{trapero2023demand} also used the Tobit
Kalman filter (TKF) for models presented in a state space framework for
forecasting purposes. This method can efficiently deal with trends,
seasonality, and exogenous influences in censored data, as long as the
forecasting model functions within a state space framework.

Certain models offer flexibility in adjusting to changes in time series
characteristics over time. Exponential Smoothing State Space models
(ETS) represent an adaptive method that has demonstrated competitiveness
in time series forecasting \autocite{gardner2006exponential}. These
approaches allow for the adaptation of model parameters over time to
accommodate shifts in time series characteristics
\autocite{hyndman2002state}. Similarly, time-varying parameter models
exhibit high adaptability \autocite{harvey2006forecasting}, enabling
forecasting models to better accommodate interruptions in data over
time. Time-varying parameters are frequently utilized in dynamic
forecasting models, such as state-space models or Bayesian structural
time series models \autocite{talih2005structural}. Some models employ
approaches to reduce the influence of affected observations during model
training by assigning different weights. The utilization of weights
extends beyond merely including or excluding observations and can also
balance the degree of influence each observation has on the forecasting
model \autocite{khoshgoftaar2007empirical}.

Disruptive events may potentially alter time series by causing gaps, or
sometimes observations can be converted into missing data. While many
time series forecasting models struggle with missing data, several
approaches can internally handle missing data \autocite{twala2008good}.
\textcite{7157837} suggested a forecasting approach based on the Least
Squares Support Vector Machine (LSSVM) that was particularly designed to
handle time series forecasting with missing data. A different study by
\textcite{tang2020joint} focused on local and global temporal dynamics
for multivariate time series forecasting with missing data. Their
proposed framework uses a memory network to capture global temporal
patterns with local data as key components.

Although several approaches exist, the literature lacks recommendations
and comparisons of alternative strategies for forecasting time series
data influenced by interruptions such as the COVID-19 pandemic. This
paper aims to close this gap by proposing several methodologies for
forecasting time series influenced by such disruptions.

\section{Handling interruptions when forecasting}\label{sec-methods}

In this section, we describe several possible strategies to handle
interruptions when forecasting time series data.

\subsection{Use a highly adaptive
model}\label{use-a-highly-adaptive-model}

Highly adaptive models can adjust to the interruption as it is happening
and will therefore be able to approximate the data generating process
relatively well. For example, an ETS model with large smoothing
parameters will be able to adjust to the interruption relatively
quickly. This has the advantage of being a very simple solution that is
easy to implement and fast to compute. There is no need to explicitly
model the interruption, and so the model can be used for forecasting
even if the timing or effect of the interruption is unknown.

However, the prediction intervals will be large because the model will
have heavily discounted past data. In fact, the model will largely
forget the past data other than the most recent observations, so there
is no memory of the seasonal patterns and other dynamics that were
present before the interruption. Consequently, the approach works best
if there is no assumption that the post-interruption period will be
similar to the pre-interruption period.

\subsection{Use an intervention model}\label{use-an-intervention-model}

A dynamic regression model with intervention covariates can be used to
model the interruption explicitly. For example, if the intervention
involves a simple level shift with a reverse level shift at the end of
the intervention, we can use a dummy variable to indicate the
interruption period and allow the model to adjust to the interruption.
More complicated interventions can be handled by using more covariates.

This has the advantage of retaining the memory of the past, and so the
seasonal patterns and other dynamics will be retained. This allows the
change period to be effectively modelled, provided the intervention
variables are chosen well. However, the model will assume that the
post-interruption period will be similar to the pre-interruption period,
and so the prediction intervals may be too narrow, especially if there
is a lasting effect beyond the end of the interruption.

\subsection{Set to missing}\label{set-to-missing}

Many time series models will handle periods of missing values. So the
problematic observations that occur during the period of disruption can
be set to missing, and the model should continue to produce forecasts as
if the interruption had not occurred. Of course, the forecasts will not
be accurate for the period of disruption, but they can be interpreted as
``what might have been''. This solution requires a judgement to be made
about when the disruption has begun, and when normality resumes.

Because no information is retained during the disruption, the prediction
intervals will become large during the disruption, and after the
disruption, they will remain large until the model has enough data to
estimate the forecast distribution more accurately.

\subsection{Estimate what might have
been}\label{estimate-what-might-have-been}

A fourth solution is to estimate what might have been during the period
of disruption and then use the adjusted data to fit a model. The
estimates could be made using any convenient method. One approach would
be to set the estimates to equal the forecasts made using only
pre-interruption data. This then becomes almost equal to the previous
solution, except that the prediction intervals will be narrower because
the estimation uncertainty has not been taken into account. On the other
hand, it is more flexible than the previous approach because models that
do not handle missing values can then be used.

An alternative would be to take the model estimated under the previous
solution (setting the observations during the disruption to missing),
and use it to estimate what might have been during the disruption. Then
the model can be re-estimated using the adjusted data. Note that
forecasts obtained in this way during the disruption will not be true
forecasts because they will have used data from the future in computing
the adjusted data during the disruption. But post-disruption forecasts
will be true forecasts and should be almost the same as those obtained
using the previous solution.

\subsection{Downweight the interruption
period}\label{downweight-the-interruption-period}

Intuitively, we want the model to be more influenced by the patterns
observed before the disruption period than those during the disruption,
but we still want to take some account of the observations during the
disruption period.

Using this weighting approach, we can leverage the insights from the
pre-disruptive period while still accounting for the disruptive period's
influence. It allows the model to better capture the underlying dynamics
of the time series in non-disrupted periods, which may be essential for
accurate forecasting in the post-disruption era.

If the weights during the disruption are set to zero and the weights at
other times are set to 1, this becomes equivalent to the missing value
solution above. Rather than ignoring the data during the disruption, the
weighting approach allows the model to not be completely blind to the
disruption's effects and adapt to changes or shifts in the time series
caused by the event.

While conceptually simple to implement, in practice, this approach may
require more work than the other methods discussed here, as most
forecasting software does not allow for the explicit use of weights.

\subsection{Ensemble models}\label{ensemble-models}

As with many forecasting problems, taking an ensemble approach often
leads to more accurate forecasts \autocite{combinations}. Here, we could
combine some or all four of the approaches discussed above to obtain an
ensemble forecast. In fact, if we were unable to implement the weighted
approach, by using an ensemble of the other approaches, we are
effectively downweighting the observations during the disruption period,
as they are only explicitly used in the first two approaches (using a
highly adaptive model or using an intervention model).

However, a disadvantage of this approach is that it is based on
averaging forecasts that aim to achieve different objectives. For
example, the highly adaptive model and the intervention model are trying
to forecast what happened during the disruption, while the missing value
and estimation approaches are trying to forecast what might have
happened. So the ensemble approach will lie somewhere between these two
objectives.

\section{Examples}\label{sec-examples}

In this section, we implement and evaluate the strategies described in
Section~\ref{sec-methods} using two daily and monthly datasets collected
over the last several years, where the impact of the COVID-19 pandemic
has been particularly evident.

\subsection{Australian tourism}\label{sec-tourism}

The Australian tourism data set is a monthly time series showing the
number of short-term overseas visitors to Australia. The data
\autocite{tourismdata} are available from January 2000 to May 2023, and
are shown in Figure~\ref{fig-tourism-plot1}. As the borders closed in
March 2020, the number of visitors to Australia dropped to near zero and
remained there until towards the end of 2021. The borders officially
reopened on 21 February 2022, although it seems visitors began to arrive
earlier than that.

\begin{figure}

\centering{

\includegraphics{fits_files/figure-pdf/fig-tourism-plot1-1.pdf}

}

\caption{\label{fig-tourism-plot1}Short-term visitor arrivals to
Australia (monthly): Jan 2000 -- May 2023.}

\end{figure}%

We apply the first four solutions discussed in the previous section to
these data, making 12 month forecasts at the end of each year from 2019
to 2022. We fit ETS, ARIMA, and dynamic ARIMA models to the data
\autocite{fpp3}, first applying a log transformation to ensure the
resulting forecasts are positive. This is possible because the
observations never reach exactly zero, with the smallest number of
visitors per month equal to 2250 in April 2020. For all forecasts, we
also show 90\% prediction intervals.

\begin{figure}[!b]

\centering{

\includegraphics{fits_files/figure-pdf/fig-tsol1-plot-1.pdf}

}

\caption{\label{fig-tsol1-plot}\textbf{Highly adaptive models}.
Forecasts from ETS and ARIMA models. Neither works particularly well for
disruptions of this magnitude.}

\end{figure}%

In Figure~\ref{fig-tsol1-plot}, we show forecasts by applying ETS and
ARIMA models to the data. ETS, in particular, is well-known to be
relatively adaptive to changes in the series, and this is evident in
these forecasts. The forecasts for 2020 were made using data before
COVID-19 had any effect, and so they show similar patterns to the past.
The forecasts for 2021 were made after 9 months of very low levels of
arrivals, and both models show forecasts consistent with recent history.
The use of logarithms is particularly important here as the variance is
much smaller during 2020 than previously, but after taking logarithms,
the variance is more stable over time. The forecasts made at the start
of 2022 have struggled to detect the small increase in traffic at the
end of 2021, and both models have forecast relatively flat trajectories
as a result, although the prediction intervals are wide, indicating
model uncertainty. Finally, the forecasts made at the end of 2022 have
captured the increasing trend, but ETS is much closer to reality,
adapting more quickly to the changing patterns. Again, the wide
prediction intervals indicate a high level of uncertainty. It is
possible to make ETS more adaptive to changes in the data by increasing
the value of the smoothing parameters. For example, a high value of
\(\beta\) (the smoothing parameter for the slope) will result in changes
in trend being incorporated into the forecasts more quickly, at the risk
of overreacting to noise in the data, and increasing the size of the
prediction intervals even more.

\begin{figure}[!b]

\centering{

\includegraphics{fits_files/figure-pdf/fig-tsol2-plot-1.pdf}

}

\caption{\label{fig-tsol2-plot}\textbf{Intervention model}. We use a
level shift from March 2020 to October 2022, and a ramp from October
2021 to October 2022. After October 2022, all intervention variables are
set to zero.}

\end{figure}%

Figure~\ref{fig-tsol2-plot} shows forecasts obtained using a dynamic
regression model (i.e., a regression with ARIMA errors) using two
intervention variables: a level shift from March 2022 to November 2022,
and a ramp from October 2021 to November 2022. It is evident that the
level shift variable has worked well, giving relatively good forecasts
for 2021. However, the forecasts for 2022 are particularly poor because
the ramp slope has been greatly overestimated, as it was based on only
three observations (Oct -- Dec 2021). The forecasts for 2023 are much
better, and the relatively large prediction intervals are appropriate
given the uncertainty in the industry at the end of 2022.

\begin{figure}[!b]

\centering{

\includegraphics{fits_files/figure-pdf/fig-tsol3-plot-1.pdf}

}

\caption{\label{fig-tsol3-plot}\textbf{Set to missing}. The period from
March 2020 to October 2022 is set to missing. So the first three years
of forecasts show what might have been. The fourth set of forecasts uses
observations from November and December 2022, and so the forecasts have
been adjusted downwards.}

\end{figure}%

The third solution (Figure~\ref{fig-tsol3-plot}) involved setting the
observations during the disruption period to missing and then fitting an
ARIMA model to the series. Consequently, the first three years of
forecasts show what might have been without the COVID-19 pandemic, based
on the history to the end of 2019. The final forecasts for 2023 use the
data to the end of 2019 and the two observations in November and
December 2022. These are much better, but the prediction intervals are
too narrow, as there hasn't been sufficient recent data.

\begin{figure}[!b]

\centering{

\includegraphics{fits_files/figure-pdf/fig-tsol4-plot-1.pdf}

}

\caption{\label{fig-tsol4-plot}\textbf{Estimate what might have been}.
We replace the observations from March 2020 to October 2022 with the
average of the same month in the three years prior to March 2020.}

\end{figure}%

We show in Figure~\ref{fig-tsol4-plot} the forecasts obtained using
solution 4. Here we have replaced observations between March 2020 and
October 2022 with estimates based on the average of the same month in
the three years prior to March 2020. The resulting forecasts are similar
to those from solution 3, but with narrower prediction intervals because
the model is (falsely) assuming that the ``observations'' during the
disruption are real.

Finally, in Figure~\ref{fig-tensemble-plot}, we show the ensemble
forecasts obtained by averaging the forecasts from the four solutions
discussed above. The first and fourth solutions carry double weight,
because both the ETS and ARIMA forecasts were included in the ensemble.
This time, no prediction intervals have been computed. The first set of
forecasts has not taken any account of the disruption, and so they are
almost identical to those obtained in the previous plots. The last set
of forecasts is reasonably good, showing what is obtained by combining
the forecasting methods used to handle the disruption period. However,
the forecasts for 2021 and 2022 are particularly poor, because they are
averaging forecasts that aim to achieve different objectives, and are
affected by the poor forecasts obtained using the dynamic regression
model.

\begin{figure}[!t]

\centering{

\includegraphics{fits_files/figure-pdf/fig-tensemble-plot-1.pdf}

}

\caption{\label{fig-tensemble-plot}\textbf{Ensemble approach}, combining
the previous four approaches shown in Figures
\ref{fig-tsol1-plot}--\ref{fig-tsol4-plot}.}

\end{figure}%

\FloatBarrier

\subsection{Example: Pedestrians}\label{example-pedestrians}

The pedestrian data set is a daily time series showing the number of
pedestrians per day in Melbourne, Australia. The data were obtained from
the Pedestrian Counting System maintained by the City of Melbourne
\autocite{pedestrians} and are based on automated hourly counts from
sensors at 66 locations over the period from 2019-01-01 to 2021-12-31.
All sensors had some missing data, and we chose to use sensors with no
more than 720 missing hours, equivalent to 30 days of missing data. This
resulted in 25 sensors being used in the analysis. The data were
aggregated to daily counts for each sensor, and then the results were
averaged across sensors to obtain a measure of pedestrian traffic per
day. The resulting time series is shown in Figure~\ref{fig-walkers}.

\begin{figure}

\centering{

\includegraphics{fits_files/figure-pdf/fig-walkers-1.pdf}

}

\caption{\label{fig-walkers}Pedestrian traffic in Melbourne, Australia,
from 2018 to 2023. The gray shaded regions indicate periods of lockdown
due to COVID-19.}

\end{figure}%

We have also shown the periods of lockdown due to COVID-19. In
Melbourne, there were six separate lockdowns ranging from 5 days to 111
days in length. The first official lockdown period was from 31 March
2020, to 12 May 2020, and the last lockdown was from 05 Aug 2021 to 21
Oct 2021. However, the first period was preceded by over a week when
most people elected to stay and work at home, so we have chosen to start
the first period on 23 Mar 2020, to better reflect human behaviour.
Otherwise, we have used the official lockdown dates
\autocite{MelbourneLockdowns}.

We evaluate the solutions discussed in Section~\ref{sec-methods} by
using time series cross-validation, with the initial training set
comprising the whole of 2019, and subsequent training sets growing by 1
week at a time, to the end of 2021. The test sets are always one week
long. Thus, we have evaluated the forecasts for 2020 and 2021, covering
all lockdown periods and the start of the recovery period. No
transformations of the data have been used. The results are shown in
Figures \ref{fig-walkers-plot1}--\ref{fig-walkers-plot4}. Prediction
intervals have not been shown to avoid cluttering the plots.

\begin{figure}

\centering{

\includegraphics{fits_files/figure-pdf/fig-walkers-plot1-1.pdf}

}

\caption{\label{fig-walkers-plot1}\textbf{Highly adaptive model}: the
forecasts do not take into account the lockdown periods, but the ARIMA
model is largely able to adjust to the changing level of the series.}

\end{figure}%

We use ARIMA models in Figure~\ref{fig-walkers-plot1}, which have mostly
been able to adapt to the changing level of the series, apart from
during the first few weeks of the first lockdown, and the first day or
two of subsequent lockdowns.

In Figure~\ref{fig-walkers-plot2}, we have used an intervention model
with a dummy variable indicating the lockdown periods. This has worked a
little better than the ARIMA model used in
Figure~\ref{fig-walkers-plot1}, apart from around the first lockdown
period. Many people were in self-imposed lockdown before the start of
the first lockdown period and were reluctant to return to the city after
it ended, resulting in poor forecasts on either side of the first
lockdown.

\begin{figure}

\centering{

\includegraphics{fits_files/figure-pdf/fig-walkers-plot2-1.pdf}

}

\caption{\label{fig-walkers-plot2}\textbf{Intervention model}: the
forecasts are based on an ARIMA model with a dummy variable indicating
when the lockdown periods occurred. Many people were in self-imposed
lockdown before the start of the first lockdown period, and were
reluctant to return to the city after it ended, resulting in poor
forecasts on either side of the first lockdown. The model was able to
adjust to the changing level of the series during subsequent lockdowns.}

\end{figure}%

\begin{figure}

\centering{

\includegraphics{fits_files/figure-pdf/fig-walkers-plot3-1.pdf}

}

\caption{\label{fig-walkers-plot3}\textbf{Set to missing}: the lockdown
periods are set to missing, and the ARIMA model uses the remaining data
to produce forecasts. Thus, the forecasts are based on only previous
non-lockdown data. Naturally, the forecasts during the lockdown periods
do not reflect reality, but they can be interpreted as what might have
been.}

\end{figure}%

\begin{figure}

\centering{

\includegraphics{fits_files/figure-pdf/fig-walkers-plot4-1.pdf}

}

\caption{\label{fig-walkers-plot4}\textbf{Estimate what might have
been}: the lockdown periods are replaced with estimated counts based on
an ARIMA model applied to the remaining data. Then a new ARIMA model is
fitted to the whole data set, and used to produce forecasts. As a
result, the forecasts during the lockdown periods are not true
forecasts, as they use non-lockdown data from the future.}

\end{figure}%

\begin{figure}

\centering{

\includegraphics{fits_files/figure-pdf/fig-walkers-ensemble-plot-1.pdf}

}

\caption{\label{fig-walkers-ensemble-plot}\textbf{Ensemble forecasts}
based on the first two solutions shown in Figures
\ref{fig-walkers-plot1} and \ref{fig-walkers-plot2}. Other than having
some difficulty around the period of the first lockdown, the results are
relatively good.}

\end{figure}%

Figure~\ref{fig-walkers-plot3} shows the results obtained by setting the
observations during the lockdown periods to missing and then fitting an
ARIMA model to the remaining data. The forecasts during lockdowns show
what might have been if that particular lockdown hadn't occurred, based
on previous non-lockdown data.

The forecasts shown in Figure~\ref{fig-walkers-plot4} are obtained by
replacing the observations during the lockdown periods with estimates
based on an ARIMA model applied to the remaining data. Then a new ARIMA
model is fitted to the whole data set, and used to produce forecasts. As
a result, the ``forecasts'' appear to interpolate across the lockdown
periods, reflecting neither the true lockdown pattern, nor the pattern
that might have occurred without the lockdowns.

Finally, in Figure~\ref{fig-walkers-ensemble-plot}, we show the ensemble
forecasts obtained by averaging the forecasts from the first two
solutions discussed above. The results are relatively good, other than
having some difficulty around the period of the first lockdown.

\section{Discussion}\label{discussion}

We have presented several approaches to forecasting in the presence of
an interruption. Each has its advantages and disadvantages, and the
choice of which to use will depend on the situation. In fact, we have
used almost all of these approaches in our own consulting work. An
intervention model is often a good solution, provided the intervention
can be modelled relatively simply. However, if the intervention is
complex, then a highly adaptive method is often better. The highly
adaptive model is also useful when the timing of the interruption is
unknown or when the post-interruption period is expected to be very
different from the pre-interruption period. The missing value approach
is particularly useful when only post-interruption forecasts are
required and forecasts during the interruption period are not needed.

We have not previously used the weighted approach in practice, and
open-source software to implement it is not currently available.
However, it seems to be a good compromise between the highly adaptive
model and the intervention model, and we expect it to be useful in many
situations once software is available to implement it easily.

Finally, the ensemble approach is useful when there is uncertainty about
which approach to use, and will often result in greater accuracy due to
the power of averaging.

We have not discussed probabilistic forecasting in any detail in this
paper, other than producing some prediction intervals for some of the
examples. One would expect forecast uncertainty to increase during, and
to some extent after, a major disruption. The highly adaptive model and
innovation model should produce relatively good prediction intervals,
although the variance is often underestimated. Because the missing value
approach is producing counterfactual (``what might have been'')
forecasts, the prediction intervals will often not include the actual
observations during the disruption period. Prediction intervals for
ensemble forecasts can be obtained by averaging the end points of the
component prediction intervals \autocite{lichtendahl2013better}, for
those models where the prediction intervals seem reasonable.

\section{Conclusion}\label{conclusion}

It is not uncommon for modellers to experience interruptions in time
series data. Both those in charge of developing forecasting models, and
managers relying on forecasts to inform decisions and policies, have
recently faced the issue of effectively forecasting time series data
influenced by the COVID-19 pandemic. While this can stand as a critical
challenge for many forecasting teams, it is important to recognise that
similar challenges could arise in circumstances other than COVID-19.
Time series can be disrupted by various factors, such as natural
disasters, policy changes, price fluctuations, definition changes,
sensor failures in the IoT (Internet of Things), maintenance, and
breakdowns in factories, just to mention a few.

The aim of this study is not to compile a comprehensive list of
forecasting models applicable to interrupted time series data. Rather,
we intend to propose overarching strategies to address such situations,
with each strategy allowing for a variety of forecasting methodologies.
As we provide publicly accessible R code, readers are encouraged to
adapt the study by employing other forecasting methods tailored to their
specific contexts.

The proposed strategies to forecast interrupted time series offer
various advantages and disadvantages. Highly adaptive models offer swift
adjustments to interruptions with ease of implementation and speedy
computation. However, this simplicity comes at the expense of wider
prediction intervals and the loss of historical pattern memory, limiting
their effectiveness in capturing pre-interruption dynamics. Intervention
models explicitly account for disruptions, retaining past patterns, yet
may yield overly narrow prediction intervals by assuming
post-interruption similarity without proper calibration. Setting
interrupted periods to missing values preserves data integrity but may
result in inaccurate forecasts during interruptions. Estimating what
might have been provides flexibility but may underestimate uncertainty.
Downweighting the interruption period balances pre- and post-disruption
data but may necessitate additional implementation complexity. Ensemble
models combine various approaches for enhanced accuracy but may
introduce challenges in interpretation.

When determining a suitable approach to forecasting interrupted time
series, several factors should be considered. To begin, the complexity
of the interruption itself is critical to approach selection.
Intervention models are a suitable approach for interruptions with known
features since they can represent the disruption explicitly. In
contrast, highly adaptable models give a simpler option for fast
adaptations to interruptions without explicit modelling, which is
especially useful when the time or effect of the interruption is
unknown. Balancing simplicity and flexibility is critical to ensuring
that the selected technique is consistent with the features of the
interruption and the required forecasting accuracy. Further, the width
of prediction intervals should be carefully assessed to determine
forecast uncertainty, particularly during and after interruptions.
Finally, for greater robustness and accuracy, investigating ensemble
models that integrate various forecasting methods might be beneficial.

Future research might focus on enhancing probabilistic forecasting
methods that are designed specifically for interrupted time series.
Large-scale empirical research evaluating the effectiveness of different
strategies in real-world settings might also provide useful insights.
Comparative studies that assess the accuracy, robustness, and computing
efficiency of different techniques across several datasets and
disruption situations can help discover the most effective approaches
for a range of practical applications.

\section*{Reproducibility}\label{reproducibility}
\addcontentsline{toc}{section}{Reproducibility}

To enhance reproducibility and facilitate the adoption of proposed
strategies, we provide the data and R code, as well as the entire paper
written in R using \href{https://quarto.org/}{Quarto} and the targets
package for R \autocite{targetsr2021}. All materials required to
reproduce this paper will be accessible via a public GitHub repository
once the paper is accepted for publication.


\printbibliography[title=References]


\end{document}
