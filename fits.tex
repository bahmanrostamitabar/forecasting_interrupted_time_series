\documentclass[11pt,a4paper,]{article}
\usepackage{lmodern}

\usepackage{amssymb,amsmath}
\usepackage{ifxetex,ifluatex}
\usepackage{fixltx2e} % provides \textsubscript
\ifnum 0\ifxetex 1\fi\ifluatex 1\fi=0 % if pdftex
  \usepackage[T1]{fontenc}
  \usepackage[utf8]{inputenc}
\else % if luatex or xelatex
  \usepackage{unicode-math}
  \defaultfontfeatures{Ligatures=TeX,Scale=MatchLowercase}
\fi
% use upquote if available, for straight quotes in verbatim environments
\IfFileExists{upquote.sty}{\usepackage{upquote}}{}
% use microtype if available
\IfFileExists{microtype.sty}{%
\usepackage[]{microtype}
\UseMicrotypeSet[protrusion]{basicmath} % disable protrusion for tt fonts
}{}
\PassOptionsToPackage{hyphens}{url} % url is loaded by hyperref
\usepackage[unicode=true]{hyperref}
\PassOptionsToPackage{usenames,dvipsnames}{color} % color is loaded by hyperref
\hypersetup{
            pdftitle={Forecasting interrupted time series},
            pdfkeywords={blah, blah},
            colorlinks=true,
            linkcolor=blue,
            citecolor=Blue,
            urlcolor=Blue,
            breaklinks=true}
\urlstyle{same}  % don't use monospace font for urls
\usepackage{geometry}
\geometry{left=2.5cm,right=2.5cm,top=2.5cm,bottom=2.5cm}
\usepackage[style=authoryear-comp,]{biblatex}
\addbibresource{references.bib}
\usepackage{longtable,booktabs}
% Fix footnotes in tables (requires footnote package)
\IfFileExists{footnote.sty}{\usepackage{footnote}\makesavenoteenv{long table}}{}
\usepackage{graphicx,grffile}
\makeatletter
\def\maxwidth{\ifdim\Gin@nat@width>\linewidth\linewidth\else\Gin@nat@width\fi}
\def\maxheight{\ifdim\Gin@nat@height>\textheight\textheight\else\Gin@nat@height\fi}
\makeatother
% Scale images if necessary, so that they will not overflow the page
% margins by default, and it is still possible to overwrite the defaults
% using explicit options in \includegraphics[width, height, ...]{}
\setkeys{Gin}{width=\maxwidth,height=\maxheight,keepaspectratio}
\IfFileExists{parskip.sty}{%
\usepackage{parskip}
}{% else
\setlength{\parindent}{0pt}
\setlength{\parskip}{6pt plus 2pt minus 1pt}
}
\setlength{\emergencystretch}{3em}  % prevent overfull lines
\providecommand{\tightlist}{%
  \setlength{\itemsep}{0pt}\setlength{\parskip}{0pt}}
\setcounter{secnumdepth}{5}

% set default figure placement to htbp
\makeatletter
\def\fps@figure{htbp}
\makeatother


\title{Forecasting interrupted time series}

%% MONASH STUFF

%% CAPTIONS
\RequirePackage{caption}
\DeclareCaptionStyle{italic}[justification=centering]
 {labelfont={bf},textfont={it},labelsep=colon}
\captionsetup[figure]{style=italic,format=hang,singlelinecheck=true}
\captionsetup[table]{style=italic,format=hang,singlelinecheck=true}

%% FONT
\RequirePackage{bera}
\RequirePackage[charter,expert]{mathdesign}
\RequirePackage[scale=0.9]{sourcecodepro}
\RequirePackage[lf,t]{FiraSans}

%% HEADERS AND FOOTERS
\RequirePackage{fancyhdr}
\pagestyle{fancy}
\rfoot{\Large\sffamily\raisebox{-0.1cm}{\textbf{\thepage}}}
\makeatletter
\lhead{\textsf{\expandafter{\@title}}}
\makeatother
\rhead{}
\cfoot{}
\setlength{\headheight}{15pt}
\renewcommand{\headrulewidth}{0.4pt}
\renewcommand{\footrulewidth}{0.4pt}
\fancypagestyle{plain}{%
\fancyhf{} % clear all header and footer fields
\fancyfoot[C]{\sffamily\thepage} % except the center
\renewcommand{\headrulewidth}{0pt}
\renewcommand{\footrulewidth}{0pt}}

%% MATHS
\RequirePackage{bm,amsmath}
\allowdisplaybreaks

%% GRAPHICS
\RequirePackage{graphicx}
\setcounter{topnumber}{2}
\setcounter{bottomnumber}{2}
\setcounter{totalnumber}{4}
\renewcommand{\topfraction}{0.85}
\renewcommand{\bottomfraction}{0.85}
\renewcommand{\textfraction}{0.15}
\renewcommand{\floatpagefraction}{0.8}

%\RequirePackage[section]{placeins}

%% SECTION TITLES
\RequirePackage[compact,sf,bf]{titlesec}
\titleformat{\section}[block]
  {\fontsize{15}{17}\bfseries\sffamily}
  {\thesection}
  {0.4em}{}
\titleformat{\subsection}[block]
  {\fontsize{12}{14}\bfseries\sffamily}
  {\thesubsection}
  {0.4em}{}
\titlespacing{\section}{0pt}{*5}{*1}
\titlespacing{\subsection}{0pt}{*2}{*0.2}


%% TITLE PAGE
\def\Date{\number\day}
\def\Month{\ifcase\month\or
 January\or February\or March\or April\or May\or June\or
 July\or August\or September\or October\or November\or December\fi}
\def\Year{\number\year}

\makeatletter
\def\wp#1{\gdef\@wp{#1}}\def\@wp{??/??}
\def\jel#1{\gdef\@jel{#1}}\def\@jel{??}
\def\showjel{{\large\textsf{\textbf{JEL classification:}}~\@jel}}
\def\nojel{\def\showjel{}}
\def\addresses#1{\gdef\@addresses{#1}}\def\@addresses{??}
\def\cover{{\sffamily\setcounter{page}{0}
        \thispagestyle{empty}
        \placefig{2}{1.5}{width=5cm}{_extensions/numbats/wp/monash2}
        \placefig{16.9}{1.5}{width=2.1cm}{_extensions/numbats/wp/MBSportrait}
        \begin{textblock}{4}(16.9,4)ISSN 1440-771X\end{textblock}
        \begin{textblock}{7}(12.7,27.9)\hfill
        \includegraphics[height=0.7cm]{_extensions/numbats/wp/AACSB}~~~
        \includegraphics[height=0.7cm]{_extensions/numbats/wp/EQUIS}~~~
        \includegraphics[height=0.7cm]{_extensions/numbats/wp/AMBA}
        \end{textblock}
        \vspace*{2cm}
        \begin{center}\Large
        Department of Econometrics and Business Statistics\\[.5cm]
        \footnotesize http://monash.edu/business/ebs/research/publications
        \end{center}\vspace{2cm}
        \begin{center}
        \fbox{\parbox{14cm}{\begin{onehalfspace}\centering\Huge\vspace*{0.3cm}
                \textsf{\textbf{\expandafter{\@title}}}\vspace{1cm}\par
                \LARGE\@author\end{onehalfspace}
        }}
        \end{center}
        \vfill
                \begin{center}\Large
                \Month~\Year\\[1cm]
                Working Paper \@wp
        \end{center}\vspace*{2cm}}}
\def\pageone{{\sffamily\setstretch{1}%
        \thispagestyle{empty}%
        \vbox to \textheight{%
        \raggedright\baselineskip=1.2cm
     {\fontsize{24.88}{30}\sffamily\textbf{\expandafter{\@title}}}
        \vspace{2cm}\par
        \hspace{1cm}\parbox{14cm}{\sffamily\large\@addresses}\vspace{1cm}\vfill
        \hspace{1cm}{\large\Date~\Month~\Year}\\[1cm]
        \hspace{1cm}\showjel\vss}}}
\def\blindtitle{{\sffamily
     \thispagestyle{plain}\raggedright\baselineskip=1.2cm
     {\fontsize{24.88}{30}\sffamily\textbf{\expandafter{\@title}}}\vspace{1cm}\par
        }}
\def\titlepage{{\cover\newpage\pageone\newpage\blindtitle}}

\def\blind{\def\titlepage{{\blindtitle}}\let\maketitle\blindtitle}
\def\titlepageonly{\def\titlepage{{\pageone\end{document}}}}
\def\nocover{\def\titlepage{{\pageone\newpage\blindtitle}}\let\maketitle\titlepage}
\let\maketitle\titlepage
\makeatother

%% SPACING
\RequirePackage{setspace}
\spacing{1.5}

%% LINE AND PAGE BREAKING
\sloppy
\clubpenalty = 10000
\widowpenalty = 10000
\brokenpenalty = 10000
\RequirePackage{microtype}

%% PARAGRAPH BREAKS
\setlength{\parskip}{1.4ex}
\setlength{\parindent}{0em}

%% HYPERLINKS
\RequirePackage{xcolor} % Needed for links
\definecolor{darkblue}{rgb}{0,0,.6}
\RequirePackage{url}

\makeatletter
\@ifpackageloaded{hyperref}{}{\RequirePackage{hyperref}}
\makeatother
\hypersetup{
     citecolor=0 0 0,
     breaklinks=true,
     bookmarksopen=true,
     bookmarksnumbered=true,
     linkcolor=darkblue,
     urlcolor=blue,
     citecolor=darkblue,
     colorlinks=true}

%% KEYWORDS
\newenvironment{keywords}{\par\vspace{0.5cm}\noindent{\sffamily\textbf{Keywords:}}}{\vspace{0.25cm}\par\hrule\vspace{0.5cm}\par}

%% ABSTRACT
\renewenvironment{abstract}{\begin{minipage}{\textwidth}\parskip=1.4ex\noindent
\hrule\vspace{0.1cm}\par{\sffamily\textbf{\abstractname}}\newline}
  {\end{minipage}}


\usepackage[T1]{fontenc}
\usepackage[utf8]{inputenc}

\usepackage[showonlyrefs]{mathtools}
\usepackage[no-weekday]{eukdate}

%% BIBLIOGRAPHY

\makeatletter
\@ifpackageloaded{biblatex}{}{\usepackage[style=authoryear-comp, backend=biber, natbib=true]{biblatex}}
\makeatother
\ExecuteBibliographyOptions{bibencoding=utf8,minnames=1,maxnames=3, maxbibnames=99,dashed=false,terseinits=true,giveninits=true,uniquename=false,uniquelist=false,doi=false, isbn=false,url=true,sortcites=false, date=year}

\DeclareFieldFormat{url}{\texttt{\url{#1}}}
\DeclareFieldFormat[article]{pages}{#1}
\DeclareFieldFormat[inproceedings]{pages}{\lowercase{pp.}#1}
\DeclareFieldFormat[incollection]{pages}{\lowercase{pp.}#1}
\DeclareFieldFormat[article]{volume}{\mkbibbold{#1}}
\DeclareFieldFormat[article]{number}{\mkbibparens{#1}}
\DeclareFieldFormat[article]{title}{\MakeCapital{#1}}
\DeclareFieldFormat[inproceedings]{title}{#1}
\DeclareFieldFormat{shorthandwidth}{#1}
% No dot before number of articles
\usepackage{xpatch}
\xpatchbibmacro{volume+number+eid}{\setunit*{\adddot}}{}{}{}
% Remove In: for an article.
\renewbibmacro{in:}{%
  \ifentrytype{article}{}{%
  \printtext{\bibstring{in}\intitlepunct}}}

\makeatletter
\DeclareDelimFormat[cbx@textcite]{nameyeardelim}{\addspace}
\makeatother
\renewcommand*{\finalnamedelim}{%
  %\ifnumgreater{\value{liststop}}{2}{\finalandcomma}{}% there really should be no funny Oxford comma business here
  \addspace\&\space}

\wp{no/yr}
\jel{C10,C14,C22}

\RequirePackage[absolute,overlay]{textpos}
\setlength{\TPHorizModule}{1cm}
\setlength{\TPVertModule}{1cm}
\def\placefig#1#2#3#4{\begin{textblock}{.1}(#1,#2)\rlap{\includegraphics[#3]{#4}}\end{textblock}}


\nocover

\author{Bahman~Rostami-Tabar, Rob J~Hyndman}
\addresses{\textbf{Bahman Rostami-Tabar}\newline
Cardiff Business School\newline
Wales CF10 3EU\newline
United Kingdom\newline
{Email: rostami-tabarb@cardiff.ac.uk}\newline Corresponding author\newline\\[0.5cm]
\textbf{Rob J Hyndman}\newline
Department of Econometrics \& Business Statistics\newline
Clayton VIC 3800\newline
Australia\newline
{Email: Rob.Hyndman@monash.edu}\\[0.5cm]
}

\date{\sf\Date~\Month~\Year}
\makeatletter
 \lfoot{\sf Rostami-Tabar, Hyndman: \@date}
\makeatother

\makeatletter
\makeatother
\makeatletter
\makeatother
\makeatletter
\@ifpackageloaded{caption}{}{\usepackage{caption}}
\AtBeginDocument{%
\ifdefined\contentsname
  \renewcommand*\contentsname{Table of contents}
\else
  \newcommand\contentsname{Table of contents}
\fi
\ifdefined\listfigurename
  \renewcommand*\listfigurename{List of Figures}
\else
  \newcommand\listfigurename{List of Figures}
\fi
\ifdefined\listtablename
  \renewcommand*\listtablename{List of Tables}
\else
  \newcommand\listtablename{List of Tables}
\fi
\ifdefined\figurename
  \renewcommand*\figurename{Figure}
\else
  \newcommand\figurename{Figure}
\fi
\ifdefined\tablename
  \renewcommand*\tablename{Table}
\else
  \newcommand\tablename{Table}
\fi
}
\@ifpackageloaded{float}{}{\usepackage{float}}
\floatstyle{ruled}
\@ifundefined{c@chapter}{\newfloat{codelisting}{h}{lop}}{\newfloat{codelisting}{h}{lop}[chapter]}
\floatname{codelisting}{Listing}
\newcommand*\listoflistings{\listof{codelisting}{List of Listings}}
\makeatother
\makeatletter
\@ifpackageloaded{caption}{}{\usepackage{caption}}
\@ifpackageloaded{subcaption}{}{\usepackage{subcaption}}
\makeatother
\makeatletter
\@ifpackageloaded{tcolorbox}{}{\usepackage[skins,breakable]{tcolorbox}}
\makeatother
\makeatletter
\@ifundefined{shadecolor}{\definecolor{shadecolor}{rgb}{.97, .97, .97}}
\makeatother
\makeatletter
\makeatother
\makeatletter
\ifdefined\Shaded\renewenvironment{Shaded}{\begin{tcolorbox}[frame hidden, sharp corners, interior hidden, enhanced, borderline west={3pt}{0pt}{shadecolor}, boxrule=0pt, breakable]}{\end{tcolorbox}}\fi
\makeatother
\makeatletter
\makeatother

% Adjust headwidth in case user has changed geometry in header-includes
\renewcommand{\headwidth}{\textwidth}

\begin{document}
\maketitle
\begin{abstract}
A brief summary of our ideas
\end{abstract}
\begin{keywords}
blah, blah
\end{keywords}

\hypertarget{introduction}{%
\section{Introduction}\label{introduction}}

Time series are sometimes interrupted by unusual events; for example, a
natural disaster may occur, or a there may be a temporary policy change.
Many forecasters have faced this issue recently with the COVID-19
pandemic, where historical patterns may have been severely disrupted due
to lockdowns and other restrictions. In this paper, we consider the
problem of forecasting after such an event has occurred.

This is a different problem from change point detection. In the
situations we consider, we know that a change has occurred, and we want
to forecast the future after the change. Change point detection is about
identifying when a change has occurred.

The changes in the series as a result of the disruption may be
relatively simple, for example a left shift at the start of the
disruption, and another at the end of the disruption. Or they may be
more complex, with changes to the seasonal patterns, and changes to the
level, which evolve over time. In this paper, we consider a range of
models that can be used to handle such changes, and compare their
performance on some real data sets.

Time series models usually assume that the future is similar to the
past, at least in how the data evolve. But a big event like covid makes
the future different from the past, at least in the short term.

\hypertarget{handling-interruptions-when-forecasting}{%
\section{Handling interruptions when
forecasting}\label{handling-interruptions-when-forecasting}}

We consider several possible ways to handle interruptions when
forecasting.

\hypertarget{use-a-highly-adaptive-model}{%
\subsection{Use a highly adaptive
model}\label{use-a-highly-adaptive-model}}

Highly adaptive models can adjust to the interruption as it is
happening, and will therefore be able to approximate the data generating
process relatively well. For example, an ETS model with large smoothing
parameters will be able to adjust to the interruption relatively
quickly. This has the advantage of being a very simple solution that is
easy to implement and fast to compute.

However, the prediction intervals will be large, because the model will
have heavily discounted past data. In fact, the model will largely
forget the past data other than the most recent observations, so there
is no memory of the seasonal patterns and other dynamics that were
present before the interruption. Consequently, the approach works best
if there is no assumption that the post-interruption period will be
similar to the pre-interruption period.

\hypertarget{use-a-dynamic-regression-model-with-intervention-covariates}{%
\subsection{Use a dynamic regression model with intervention
covariates}\label{use-a-dynamic-regression-model-with-intervention-covariates}}

A dynamic regression model with intervention covariates can be used to
model the interruption explicitly. For example, if the intervention
involves a simple level shift, with a reverse level shift at the end of
the intervention, we can use a dummy variable to indicate the
interruption period, and allow the model to adjust to the interruption.
More complicated interventions can be handled by using more covariates.

This has the advantage of retaining the memory of the past, and so the
seasonal patterns and other dynamics will be retained. However, the
model will assume that the post-interruption period will be similar to
the pre-interruption period, and so the prediction intervals may be be
too narrow.

Advantages: retains full memory of the past, and allows the change
period to be effectively modelled provided you choose the covariates
well.

Disadvantages: requires a lot of thought to choose the covariates well.
Assumes that the post-pandemic period will be similar to the
pre-pandemic period.

\hypertarget{treat-the-covid-period-as-missing-and-use-a-model-that-handles-missing-values}{%
\subsection{Treat the covid period as missing and use a model that
handles missing
values}\label{treat-the-covid-period-as-missing-and-use-a-model-that-handles-missing-values}}

Advantages: effectively ignores the change period

Disadvantages: throws away the covid-period data, and so starts the
post-covid period with no recent history. Therefore prediction intervals
will be relatively large.

\hypertarget{estimate-what-might-have-been-and-adjust-the-data}{%
\subsection{Estimate what might have been and adjust the
data}\label{estimate-what-might-have-been-and-adjust-the-data}}

Advantages: relatively simple and doesn't throw away the covid-period
data.

Disadvantages: need to have good estimates of covid-period data, and
that may be difficult to obtain. Forecasts are conditional on the
estimates being correct.

\hypertarget{examples}{%
\section{Examples}\label{examples}}

\hypertarget{example-ambulance-attendances}{%
\subsection{Example: Ambulance
attendances}\label{example-ambulance-attendances}}

\includegraphics{fits_files/figure-pdf/plot-data-1.pdf}

\includegraphics{fits_files/figure-pdf/plot-data-2.pdf}

\includegraphics{fits_files/figure-pdf/unnamed-chunk-3-1.pdf}

\includegraphics{fits_files/figure-pdf/forecast-1.pdf}

\includegraphics{fits_files/figure-pdf/forecast-2.pdf}

\includegraphics{fits_files/figure-pdf/forecast-3.pdf}

\printbibliography

\end{document}
